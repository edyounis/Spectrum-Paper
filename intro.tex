\section{Introduction}
Ever since the disclosure of Spectre and Meltdown at the beginning of 2018, there has been a large number of attacks targeting speculative microarchitectural state in out-of-order processors [cite].%
%
Each of these attacks is able to exploit the shared state of the processor between a victim process/thread and a malicious process/thread.%
%
While classifications have been developed for cache-based and for timing-based side-channel attacks [cite], a general taxonomy for the space of speculative execution attacks has not been formalized.%
%
Currently, these attacks have only been broadly identified as variants of Spectre.%
%
A well-devised taxonomy would inform not only the categorization of new attacks but also the development of hardware mitigations for these attacks.%
%
A complete categorization system would help hardware researchers identify general mitigations for broad categories of vulnerabilities, instead of targeted fixes for specific attacks.%
%

Additionally, the BOOM [cite] open-source RISC-V microarchitecture has not been exposed to these types of attacks yet.%
%
BOOM (The Berkeley Out-of-Order Machine) is a generic implementation of a speculative out-of-order microarchitecture that employs many of the same features as the proprietary systems targeted by published attacks.%
%
Generic, open-source implementations of attacks and their mitigation strategies would provide a common foundation for the development of defenses for speculative execution attacks.%
%

By attacking an open-source machine with Spectre, researchers can not only gain more insight into the malicious attack but also attempt to mitigate the attack on a shared, open platform.%
%
Moreover, this project is hoped to unify and guide the attack mitigation process.%
%
With an attack taxonomy and an open-source microarchitecture, a newly proposed mitigation technique can easily target a class of attacks as informed by the taxonomy.

