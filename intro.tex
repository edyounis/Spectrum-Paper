\section{Introduction}

Ever since the disclosure of Spectre and Meltdown at the start of 2018, there has
been a large number of attacks targeting speculative microarchitectural state in
out-of-order processors \cite{b1,b2}. These attacks are able to exploit the shared
state of the processor between a victim process/thread and a malicious process/thread.
While classifications have been developed for speculative attacks 
\cite{b5,b6,b7,b9,b10}, a general taxonomy for these attacks that is has not been adopted.
A well-devised taxonomy would inform not only the 
categorization of new attacks but also the development of hardware mitigations for 
these attacks. A complete categorization system would help hardware researchers 
identify general mitigations for broad categories of vulnerabilities, instead of 
targeted fixes for specific attacks.

Additionally, the Berkeley Out-of-Order Machine (BOOM) open-source RISC-V
microarchitecture has yet not been exposed to these types of attacks \cite{b11}. BOOM
is a generic implementation of a speculative 
out-of-order microarchitecture that employs many of the same features as the 
out-of-order systems targeted by published attacks. Generic, open-source 
implementations of attacks and their mitigation strategies would provide a common 
foundation for the development of defenses for speculative execution attacks.
By attacking an open-source machine with Spectre, researchers can not only gain more 
insight into the malicious attack but also attempt to mitigate the attack on a shared,
open platform. Moreover, this project hopes to unify and guide the attack 
mitigation process. With an attack taxonomy and an open-source microarchitecture, a 
newly proposed mitigation technique can easily target a class of attacks as informed 
by the taxonomy.

The remainder of this paper is structured as follows: In Section \ref{Related Works}, we discuss
related works that indicate relevant work that introduces attacks, defenses, and classifications.
In Section \ref{Proposed Taxonomy} we discuss our taxonomy and it's categorization of the attacks and defenses. In
Section \ref{Speculative Attack Replication} we present the subset of attacks that were replicated for this paper. In Section \ref{Speculation Buffer}
we discuss the implementation of the SpecBuf. In Section \ref{Evaluations} we discuss the evaluation of the
replications and the SpecBuf. In Section \ref{Future Work} we discuss future work.
