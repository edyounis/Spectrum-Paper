\begin{abstract}
    %TODO: does the last sentence sound good

    The discovery of exploits targeting the effects of speculative execution has
    forced the computer architecture community to rethink the design space of
    high-performance and secure microarchitectures. This new category of attacks,
    broadly categorized as ``Spectre'', has spurred substantial research on how to
    mitigate vulnerabilities associated with high-performance microarchitectural
    features. In this paper, we present three results to advance research on
    secure hardware mitigations for these threats; a taxonomy for speculative style
    attacks and defenses, replication of speculative attacks on an open-source processor,
    and a hardware implementation of a speculative buffer cache that mitigates
    basic speculative cache attacks.
    
    We have categorized the variety of attacks based on if the attacks are 
    architecturally legal and the defenses based on their types. Additionally, 
    we have mapped mitigations to the attacks that they potentially cover.
    Using the Berkeley Out-Of-Order Machine (BOOM), we have 
    re-implemented Spectre Variants 1 and 2 that target the L1 Data Cache. We 
    achieve a leakage rate of around 114 bytes per sec with a 100 MHz processor 
    frequency. Additionally, we propose and implement a speculative buffer,
    called SpecBuf, that holds speculative data in the MSHRs until commit where the 
    data will be transfered to the L1. Based on preliminary results, it achieves 
    a 2216 Dhrystones/s score on the Dhrystone benchmark
    resulting in a 2\% performance increase compared to the
    baseline. Additionally, trial synthesis on a 45nm process shows a 2.5\% area
    increase and 0.36\% clock reduction. To our knowledge, these replications and 
    SpecBuf implementation is the first demonstration of a speculative style attack
    on BOOM and the first synthesizable RTL implementation of a speculative buffer.
\end{abstract}
