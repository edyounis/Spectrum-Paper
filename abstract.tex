\begin{abstract}
    With the discovery of speculative execution exploitative attacks the computer
    architecture community is rethinking the architecture boundary and how the
    attacks can be mitigated. This project aims to tackle this new set of problems
    by providing three different outcomes; a comprehensive taxonomy for attacks
    and defenses, replication of speculative attacks on an open-source processor,
    and a mitigation developed for some of these attacks on an open-source
    processor.
    
    We have categorized the variety of attacks based on if the attacks are 
    architecturally legal and the defenses based on their types. Additionally, 
    for the taxonomy we have mapped mitigations to the proper attacks that they 
    potentially cover. Using the Berkeley Out-Of-Order Machine, we have 
    re-implemented Spectre Variants 1 and 2 that target the L1 Data Cache. We 
    achieve a leakage rate of around 114 bytes per sec with a 100MHz processor 
    frequency. For mitigating these speculative load type of attacks, we 
    implemented a speculative buffer that resides within the MSHRs. Based on
    preliminary results, it achieves a 2151 score on the Dhrystone benchmark
    resulting in a 1\% performance decrease compared to the baseline. Additionally,
    synthesizing around a 45nm process shows a 3\% area increase and 3\% lower clock
    impact.
\end{abstract}
