\section{Related Works}

\subsection{Speculative Style Attacks}

\subsubsection{Spectre}

This work first described a speculative style attack on out-of-order cores that 
allows a malicious user to read secret data from a victim \cite{b1}. When an out-of-order 
processor reaches a control flow instruction, it uses the Branch Target Buffer in 
conjunction with the Branch Predictor to make an educated guess on the direction
of program flow. If the control flow is mispredicted then the code that was
executed will not be committed into the architectural state. However, the
speculated instruction can changed the microarchitectural state causing data
leakages. This work introduced two main variants that leak secret data through
a cache side channel attack. The first variant named the Bounds Check Bypass
trains the Branch Predictor to ignore a bounds check and retrieve secret data. The
second variant uses the Branch Target Buffer to determine which target to execute
called a gadget. Although both attacks are simple exploits of speculative execution,
this attack affects all types of out-of-order speculative microarchitectures 
and requires significant changes to the microarchitecture and software stack to solve.
Our work serves as an extension of this work by categorizing, replicating, and mitigating
speculative cache side channel attacks on an open source out-of-order microarchitecture.

\subsubsection{Meltdown}

This work describes another form of speculative attack attack where an attacker 
can bypass privilege levels and read all the data from kernel memory \cite{b2}. Exploiting a 
similar mechanism to Spectre, this attack uses speculated code run after memory access
protection exception to leak kernel memory to the attacker. When the memory access
is run, then the privilege check occurs in parallel with retrieving the data. Thus,
for this attack to work, the data is retrieved and is used to modify the 
microarchitectural state before instruction causes an exception. This type of attack
is more dangerous than Spectre in that it leaks kernel memory (and physical memory if
it is mapped in the kernel), however, it is easier to mitigate through mechanisms such
as KAISER \cite{b53} which unmaps the physical memory from the kernel preventing the attack.
Our work for this paper, addresses the leakage mechanism of this particular attack but 
does not solve the source of the attack since it is specific to x86 architectures.

\subsection{Speculation Defenses}

\subsubsection{InvisiSpec}

InvisiSpec is a theoretical implementation of a speculative buffer on a multi-core
processor \cite{b46}. This buffer is designed close to the processor and holds all unsafe speculative loads.
Assuming infinite area and energy, this is implemented by extending the Load Queue (LQ).
InvisiSpec's major point is that its buffer is invisible to memory hierarchy, and as a result,
invisible to the cache coherence protocol. This implies that the hardware must track cache
coherence requests that effect speculated loads and apply them correctly and safely. This is
done with an exposure scheme that enforces memory consistency and cache coherence protocols.
InvisiSpec achieves a 72\% slowdown when simulating with Gem5. Our implementation only
considers a single-core processor scoping out the cache coherence protocol. However, we consider
a multi-core processor, no change would be needed to the cache coherence because we treat the
speculative data as part of the state of the cache. Additionally, by extended the Miss
Status Holding Registers (MSHR) instead of the LQ we achieve practical results in terms
of physical area and energy.

\subsubsection{SafeSpec}

SafeSpec is another theoretical implementation of a speculative buffer on an x86 processor \cite{b29}.
Unlike InvisiSpec and our work, SafeSpec considers adding speculative buffers to the TLB and 
i-cache as well as the d-cache. This results in significant control flow changes.
Like InvisiSpec, SafeSpec is not practically concerned with physical limitations.
Their simulated CPU has an 26.4\% power and 17\% area overhead in the worst case. One
important contribution of SafeSpec is the paper's discussion on side-channels using the speculative
buffer. In essence, it is a few orders of magnitude more difficult and inefficient to perform
these side-channels. We are only concerned with protecting the data-cache from side-channels, as
there are many modules on the processor already that present other possible side-channels.[cite needed?]

\subsection{Other Classifications}

\subsubsection{A systematic Evaluation of Transient Execution Attacks and Defenses}

This work proposes a systematic classifications for speculative attacks \cite{b48}. While categorizing
the attacks, the authors also systematically discovered a few variants of these types of attacks.
This paper focuses on the effectiveness of attacks and defenses on different CPU vendors, where as,
we stay agnostic to the CPU manufacturer. Additionally, this paper does a in-depth analysis of Spectre-type
mitigations, but has a very short evaluation of meltdown defenses.
