\section{Related Works} \label{Related Works}

\subsection{Speculative Style Attacks}

\subsubsection{Spectre}

This work first described speculative style attacks on out-of-order cores that 
allow a malicious user to read secret data from a victim \cite{b1}. When an out-of-order 
processor reaches a control flow instruction, it predicts the direction of program flow
using the Branch Target Buffer and Branch Predictor.
If the control flow is mispredicted, then the code that was
executed will not be committed into the architectural state. However, data leakages can 
occur due to the speculated instruction changing the microarchitectural state.
This work introduced two main variants that leak secret data through
a cache side channel attack. The first variant named Bounds Check Bypass
trains the Branch Predictor to ignore a bounds check and retrieve secret data. The
second variant uses the Branch Target Buffer to determine which target to execute
called a gadget. Both attacks exploit speculative execution and thus
affect all types of out-of-order speculative microarchitectures.
Therefore, significant changes to the microarchitecture and software stack will have
to occur to mitigate these attacks.
Our work serves as an extension of this result by categorizing, replicating, and mitigating
speculative cache side channel attacks on an open source out-of-order microarchitecture.

\subsubsection{Meltdown}

This work describes another speculative attack where an attacker 
can bypass privilege levels and read all the data from kernel memory \cite{b2}.
This attack uses speculated code run after a memory access
protection exception to leak kernel memory to the attacker. When a kernel memory 
location is accessed in a lower privilege level, then the privilege check occurs
in parallel with retrieving the data. Thus, when this data is retrieved, the following
instructions can be speculatively executed, leaking the information through a side
channel. This type of attack
is more dangerous than Spectre in that it leaks kernel memory (and physical memory if
it is mapped in the kernel), however, it is easier to mitigate through mechanisms such
as KAISER \cite{b21} which unmaps the physical memory from the kernel preventing the attack.
Our work for this paper addresses the leakage mechanism of this particular attack but 
does not solve the source of the attack since it is specific to x86 architectures.

\subsection{Speculation Defenses}

\subsubsection{InvisiSpec}

InvisiSpec is a theoretical implementation of a speculative buffer on a multi-core
processor \cite{b46}. This buffer is designed to extend the Load Queue (LQ) and hold all unsafe speculative loads.
Since InvisiSpec's buffer is invisible to the memory hierarchy, the buffer is
invisible to the cache coherence protocol. This implies that the hardware must track cache
coherence requests that effect speculated loads and apply them correctly and safely. This is
done with an exposure scheme that enforces memory consistency and cache coherence protocols.
InvisiSpec achieves a 72\% slowdown during software simulation. Our implementation differs by only
considering a single-core processor which allows us to ignore cache coherence. Additionally, we extended the Miss
Status Holding Registers (MSHRs) instead of the LQ to achieve practical and synthesizable results
rather than InvisiSpec's buffer that assumes infinite area and size.

\subsubsection{SafeSpec}

SafeSpec is another theoretical implementation of a speculative buffer on an x86 processor \cite{b29}.
Unlike InvisiSpec and our work, SafeSpec considers adding speculative buffers to the Translation Lookaside Buffer and 
L1 caches which results in significant control flow changes.
Simulated CPU results show an 26.4\% power and 17\% area overhead in the worst case. One
important contribution of SafeSpec is the paper's discussion on side channels using the speculative
buffer. With the buffer, the side channels are a few orders of magnitude more difficult and inefficient to perform.
We differ from this work by creating a working implementation of a speculative buffer that protects the data cache.

\subsection{Other Classifications}

\subsubsection{A Systematic Evaluation of Transient Execution Attacks and Defenses}

This work proposes a systematic classification for speculative attacks \cite{b48}. 
In addition to categorizing the attacks, the authors also described other undiscovered speculative attack variants.
Additionally, this paper does a in-depth analysis of Spectre-type mitigations, but has a very short evaluation of Meltdown defenses.
Our taxonomy differs from this work by mainly focusing on a vendor agnostic solutions to speculative side channel attacks.
