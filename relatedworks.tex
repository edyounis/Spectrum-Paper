\section{Related Works}

\subsection{Speculative Style Attacks}

\subsubsection{Spectre}

This work first described a speculative style attack on out-of-order cores that 
allows a malicious user to read secret data from a victim. When an out-of-order 
processor reaches a control flow instruction, it uses the Branch Target Buffer in 
conjunction with the Branch Predictor to make an educated guess on the direction
of program flow. If the control flow is mispredicted then the code that was
executed will not be committed into the architectural state. However, the
speculated instruction can changed the microarchitectural state causing data
leakages. This work introduced two main variants that leak secret data through
a cache side channel attack. The first variant named the Bounds Check Bypass
trains the Branch Predictor to ignore a bounds check and retrieve secret data. The
second variant uses the Branch Target Buffer to determine which target to execute
called a gadget. Although both attacks are simple exploits of speculative execution,
this attack affects all types of out-of-order speculative microarchitectures 
and requires significant changes to the microarchitecture and software stack to solve.
Our work serves as an extension of this work by categorizing, replicating, and mitigating
speculative cache side channel attacks on an open source out-of-order microarchitecture.

\subsubsection{Meltdown}

This work describes another form of speculative attack attack where an attacker 
can bypass privilege levels and read all the data from kernel memory. Exploiting a 
similar mechanism to Spectre, this attack uses speculated code run after memory access
protection exception to leak kernel memory to the attacker. When the memory access
is run, then the privilege check occurs in parallel with retrieving the data. Thus,
for this attack to work, the data is retrieved and is used to modify the 
microarchitectural state before instruction causes an exception. This type of attack
is more dangerous than Spectre in that it leaks kernel memory (and physical memory if
it is mapped in the kernel), however, it is easier to mitigate through mechanisms such
as KAISER [?] which unmaps the physical memory from the kernel preventing the attack.
Our work for this paper, addresses the leakage mechanism of this particular attack but 
does not solve the source of the attack since it is specific to x86 architectures.

\subsection{Speculation Defenses}
\subsubsection{InvisiSpec}
\subsubsection{SafeSpec}

\subsection{Other Classifications}
\subsubsection{A systematic Evaluation of Transient Execution Attacks and Defenses}
