\section{Future Work}

\subsection{Bounds Check Bypass and Branch Target Injection Improvements}

While the replicated attacks are a PoC of the speculative type of attacks that BOOM
is susceptible to, future work for this can focus on speeding up the attacks. 

One of the mechanisms to speed up both attacks is the hand made L1 cache flush function
created. The current implementation uses a large array to index into to evict a particular
set in the cache. Another approach is to have just the L1 cache size amount of data and use 
arithmetic to figure out where in this smaller block to access (note: remember that when
you have an array statically allocated, it doesn't have to start at set 0). Additionally,
another approach to speeding up the function is to prespecify the addresses that need to be
accessed to clear the cache and use that to clear the set. 

In addition to fixing the access of the array in the cache flush function, another
improvement can be to tweak the different parameters of the attack. This includes
both the attack parameters listed in \ref{tab:attack-params} but also the amount of
hits to the same set in the cache function. Potentially, the amount of rounds on the
same bytes and training rounds can be reduced to improve the speed of the attack.
These can be adjusted in conjunction with the cache flush hits on the same set so even though the 
random cache replacement policy may not clear all the ways of the set, but the amount of rounds on
the same byte might remove the false positives.

\subsection{Other Attacks}

In addition to the two attacks that were implemented, BOOM is susceptible to a variety of
other speculative style attacks. Future work would be to replicate these attacks on BOOM.
One such example is the RSB attack, our team was originally planning on implementing this
attack but BOOM did not have a working version with the RSB at the time. Thus, by fixing 
the RSB, the team could work on implementing this new attack with the code that was initially
created for this attack. Additionally, the team can train this attack on multiple different
types of branch predictors. One of the branch predictors that was implemented in BOOM was
the TAGE BP. 

\subsection{Better evaluations}
use hpms

more benchmarks

better synth - use sram

\subsection{BOOM Improvements}

Future work for the project also involves bringing BOOM to a more stable version. Throughout
the project, a multitude of bugs were found both in the front-end (fetch, bpu, etc) and the 
load/store unit that the team had to work around. With these fixed, not only will performance
of the core improve, but other replication and modifications to the core will be easier since
the base core will be stable.

multiported cache

specboom improvements - fully associatibe

