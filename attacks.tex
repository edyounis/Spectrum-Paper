\section{Speculative Attack Replication} \label{Speculative Attack Replication}

To our knowledge, speculative attacks have not yet been implemented on BOOM. Thus,
one of the goals of this project was to demonstrate simple proof of concept implementations for a
subset of speculative style attacks. Originally, the goal was to implement both 
Spectre-v1, otherwise know as the Bounds Check Bypass attack, and the Return Stack Buffer attack \cite{b3}.
However, due to the instability of BOOM's high-performance branch predictors, the team instead implemented
proof of concepts for Bounds Check Bypass attack and Spectre-v2, otherwise known as the
Branch Target Injection attack. Over 520 lines of C code were written for both implemented attacks as well as 
code for the untested Return Stack Buffer attack. The source code for these replicated attacks is available at
\url{https://github.com/riscv-boom/boom-attacks}.

\subsection{Implementation Details}

This project utilizes the Berkeley Out-of-Order Machine (BOOM), a synthesizable and 
parameterizable open source RV64G RISC-V core written in the Chisel hardware construction language \cite{b49}. 
The microarchitecture is open source, providing insight into the front-end and memory system
used. The ability to introspect on the inner mechanics of the core enabled us to quickly replicate 
the attacks and maximize their performance. We describe some of the issues encountered in this process:

\subsubsection{Flushing a cache line}

One feature of the popular x86 and ARM ISAs is the availability of cache manipulation instructions
(for example x86 has {\tt clflush} to flush a cache line from every level of the cache hierarchy).
However, RISC-V does not have a corresponding {\tt clflush} equivalent.
This type of instruction is extremely important for cache side channel attacks since it allows for
the attacker to precisely control the contents of the cache.
Therefore, a workaround was made to implement an equivalent
L1 cache line flush function. This function takes in an address and size in bytes, and flushes
all sets starting from that address until the size. This is done by having an extra array which 
is a multiple of the L1 cache size, that when indexed into, evicts a cache line within the set that the input
address refers to.

While {\tt clflush} is able to selectively remove a cache line from the cache without 
affecting the rest of the cache state, the function created is only able to evict the entire set.
This enforces that the cache line referenced is evicted at the cost of evicting other lines, causing a performance
decrease. Evicting the entire set is done by accessing the extra array addresses corresponding to the set 
{\tt N * L1\_WAY} times, where {\tt N} is large enough to account for the random cache replacement policy. The larger
the {\tt N} value, the higher probability that the particular cache line is evicted from the set. For our purposes, this value
was chosen to be {\tt N = 4} in order to have a 99\% probability of evicting the referenced cache line.

\subsubsection{BOOM Branch Predictor Unit}

% TODO: double check that the ghr is hashed and folded

The BOOM Branch Predictor Unit (BPU), is split into a single-cycle "next-line predictor" (NLP) and a
slower but more complicated "backing predictor" (BP). The NLP contains the Branch Target Buffer (BTB) 
where more recent branches are cached alongside their target, a tag of the PC, and more metadata. Additionally,
the NLP contains the Return Stack Buffer (RSB) which holds the return pointer for a {\tt ret} instruction. The BP
consists of a large predictor such as a TAGE or GShare predictor which makes a more accurate prediction based on
a larger view of the branch history \cite{b47}. For simplicity, the team used a GShare predictor, since it is 
easier to train and was fully functioning at the start of the project. The GShare predictor had a global history of 23 bits and history table containing 4096 
counters. Since the global history was longer than the size of the table, the global history was hashed with the 
instruction PC and folded to index into the history table. Additionally, when trying to
implement the RSB attack, it was discovered that the RSB was disabled causing all {\tt ret} instructions
to speculate to the next PC instead of the return target PC. Re-enabling the RSB led to multiple other BPU bugs,
so we decided to keep the RSB disabled and instead implement the Branch Target Injection attack.

\subsubsection{BOOM Memory System}

% TODO: does this make sense to people

BOOM's memory system utilizes the RocketChip L1 Data Cache and connects to the rest of the RocketChip
uncore \cite{b54}. However, in the configuration of BOOM used, there is a L1 cache and outer memory set to a L2 cache
latency. This memory hierarchy still functions properly, however it was noticed that the speculation window
was not large enough due to the memory model mandating that the latency to any memory request would be no greater than latency to L2.
While a real attack would use L2 or L3 cache misses to expand the speculation window, we instead approximated this with a {\tt fdiv} dependency chain.

\subsection{Bounds Check Bypass Attack} \label{Bounds Check Bypass Attack}

In this attack, the attacker resides in the same address space as the victim and uses
conditional branches to read victim data. The attacker would first train a conditional branch
within the victim code to predict the branch to be taken in a certain direction. Then,
during the attack round, the attacker
gives a value to the victim that fails the bounds check in the victim but during
speculation is used to retrieve the secret value. This particular attack variant
uses a cache side channel attack to leak the information by having a passed in value retrieve
a secret value which is then used to index the "attacker array". After the attack is run, this 
"attacker array" can be scanned and timed to determine the secret value based on the timings
of cache hits. The psuedocode for this type of attack in Code \ref{code:spectre-v1-pseudo}
demonstrates the attack concept using a C-like syntax.

\begin{lstlisting}[style=column-code, label={code:spectre-v1-pseudo}, caption=Psuedocode of Bounds Check Bypass Attack]
char* secret[] = "SECRET_VALUE"
char  attackArray[256 * CACHE_BLOCK_SZ]

void victim(idx){
  if (idx < secret.size())
    attackArray[secret[idx] * CACHE_BLOCK_SZ]
}

int main(void){
  // Train BPU and Attack secret value
  while (TRAINING_ROUNDS)
    victim(idx)

  victim(outOfBoundsIdx)

  // Probe array to check for hit
  for (i -> 0 until 256){
    t0 = time()
    attackArray[i * CACHE_BLOCK_SZ]
    t1 = time()
  }
}
\end{lstlisting}

% TODO: double check the bpu hashing

The BOOM implementation of this attack echos the Spectre-v1 proof of concept and the psuedocode closely.
It consists of three different parts. First, a tally buffer is cleared before reading a secret byte.
This tally buffer is used to keep track of the amount of hits a secret value gets during the measurement
phase. Next the array to be probed is evicted from the cache (the "attacker array" mentioned above), and
the attacker trains the BPU to enter the bounds protected area within the victim function. Note that in
this section, there is an extra loop that puts the branch history in a consistent state since the branch
history is hashed on update. Finally, the attacker code reads out the probing array through using the
{\tt rdcycle} instruction and checks if the result is under a hand tuned cache threshold. Full code for 
this attack is at Code \ref{code:boom-spectre-v1}.

\subsection{Branch Target Injection Attack}

This attack exploits the outcome
of {\tt jalr} instructions which use the BTB to predict the destination. In this attack,
the attacker trains a BTB entry for a {\tt jalr} instruction to point to 
the victim code. Then during the attack run, when the {\tt jalr} is supposed to go to another function,
the BTB redirects the PC to the victim code where it speculatively leaks the secret value. Similar to the
Bounds Check Bypass attack, this attack leaks information through the
cache side channel mentioned in Section \ref{Bounds Check Bypass Attack}. The psuedocode in 
Code \ref{code:spectre-v2-pseudo} demonstrates the basic principle of the attack using a C-like syntax.

\begin{lstlisting}[style=column-code, label={code:spectre-v2-pseudo}, caption=Psuedocode of Bounds Check Bypass Attack]
char* secret[] = "SECRET_VALUE"
char  attackArray[256 * CACHE_BLOCK_SZ]

void otherFunction(void){
  // Other functionality    
}

void victim(idx){ attackArray[secret[idx] * CACHE_BLOCK_SZ] }

int main(void){
  
  // Training
  while (TRAINING_ROUNDS){
    jump(victim)
  }

  // Attack. Read secret value
  argument = attackIdx
  jump(otherFunction)

  // Probe array to check for hit
  for (i -> 0 until 256){
    t0 = time()
    attackArray[i * CACHE_BLOCK_SZ]
    t1 = time()
  }
}
\end{lstlisting}

The BOOM implementation of this attack is very similar to the implementation given for
the Bounds Check Bypass attack listed in Section \ref{Bounds Check Bypass Attack}. One of the main differences
for this attack is that the attacker has to change both the address of the function being called through
{\tt jalr} and the passed in value to the victim function throughout the training and the attack run.
Another difference between the two attacks is that the {\tt fdiv} manipulation is instead done on the target address
in the main function instead of using {\tt fdiv} on the passed in index in the victim code.
