\section{Proposed Taxonomy}
In this section we discuss our classification of speculative attacks and mitigations in Figure \ref{fig:categorization}. %
This taxonomy was built from the perspective of the architecture. We only considered a single-processor case. %
Our threat model includes only an adversary who can arbitrarly read data from the cache and write speculative data to the cache. %

For Attacks we did our top level split based on whether the attack is architecturally a legal action or not. %
In a situation where an illegal action was performed a fault is raised. %
The different types of attacks in this section are associated with different types of faults. %
The Page Fault, however, has many different attacks associated with it that only differ on the permission bit abused in the attack. %
This becomes another split in attack tree. %
The Spectre-Type attacks that abuse an architecturally legal action all target different hardware modules. %
This forms the split for that branch of the tree. %

Defenses are organized according to implementation location and cost. %
At the bottom of the stack we have the hardware mitigations, which are further divided into new modules or changed modules. %
Changed modules imply a change to the function of an existing hardware component, but a new module is an entirely new hardware idea. %
Firmware was seperated into it's own category because the implementation effort is drastically different than the implementation of hardware or software. %
At the top, software defenses are split only into two categories, OS and Application level mitigations. %
This is done because operating systems cannot always be trusted nor can applications. %
Depending on the users threat model, a conclusion can be reached on which mitigations to assume or implement. %

\begin{center}
\begin{figure}[!p]
    \makebox[\textwidth]{\includegraphics[width=1.1\paperwidth,angle=270]{categorization}}
    %\captionsetup{justification=centering}
    \caption{Full taxonomy of different attacks and defenses and what they cover}
    \label{fig:categorization}
\end{figure}
\afterpage{\clearpage}
\end{center}
